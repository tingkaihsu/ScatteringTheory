\documentclass[12pt]{article}
\usepackage{braket}
\usepackage{physics}
\usepackage{graphicx}
\usepackage{times}
\usepackage[export]{adjustbox}
\usepackage{listings}
\usepackage{mathcomp}
\usepackage{hyperref}
\usepackage{bm,amsmath}
\usepackage{amssymb}
\usepackage{float}
\usepackage{indentfirst}
\usepackage{bigints}
\usepackage{listings}
\usepackage{color}
\hypersetup{
colorlinks=true,
linkcolor=blue,
filecolor=magenta,
urlcolor=cyan,
pdftitle={Overleaf Example},
pdfpagemode=FullScreen,
}
\definecolor{dkgreen}{rgb}{0,0.6,0}
\definecolor{gray}{rgb}{0.5,0.5,0.5}
\definecolor{mauve}{rgb}{0.58,0,0.82}
\lstset{frame=tb,
language=Python,
aboveskip=3mm,
belowskip=3mm,
stepnumber = 1,
showstringspaces=false,
columns=flexible,
basicstyle={\small\ttfamily},
numbers=left,
numberstyle=\color{gray},
keywordstyle=\color{blue},
commentstyle=\color{dkgreen},
stringstyle=\color{mauve},
breaklines=true,
breakatwhitespace=true,
tabsize=3
}
\numberwithin{equation}{section}

\title{Scattering Theory}
\author{Ting-Kai Hsu}
\date{\today}

\begin{document}
\maketitle
\tableofcontents
This article is designed for leading the reader from old scattering theory of quantum mechanics to nowadays scattering theory of quantum field theory.
\section{Potential Scattering}
In this chapter we would study the theory of scattering in a simple but important case, the elastic scattering of a non-relativistic particle in a local potential, but using modern techniques that could easily be extended to more general problems.
\subsection{In-States}
Consider a non-relativistic particle of mass $\mu$ in a potential $V(\mathbf{x})$, whose Hamiltonian is,
\begin{equation}
    H = H_0 + V(\mathbf{x})
\end{equation}
Here $H_0 = \mathbf{p}^2/2\mu$ is the kinetic energy operator. 
The potential is assumed to be function of position operator, and tends to zero as $|\mathbf{x}| = r\rightarrow\infty$.
Then it is concerned here with a positive-energy particle\footnote{Actually, there could possibly be negative energy if it is in bound state, that is, the energy is less than potential at boundaries. But now, we are considering scattering state, which is opposite with bound state.}.
The particle comes into facing the potential from great distances regarded to have no interaction with momentum $\hbar\mathbf{k}$, and is scattered, going out again to infinity, generally along a different direction.
\\\indent In Heisenberg picture, for a particle with momentum $\hbar\mathbf{k}$ far from the scattering center if the measurement of the particle is made at very early times, and this situation is represented by a time-independent state vector $\ket*{\Psi^{\text{in}}_{\mathbf{k}}}$.
As mentioned, at very early times the particle is at a location that is far from the scattering center so that the potential is negligible there, so the state has an energy $\hbar^2\mathbf{k}^2/2\mu$, with the following relation,
\begin{equation}
    H\ket*{\Psi_{\mathbf{k}}^{\text{in}}} = \frac{\hbar^2\mathbf{k}^2}{2\mu}\ket*{\Psi_{\mathbf{k}}^{\text{in}}}
\end{equation}

Now switch to Schr$\ddot{\text{o}}$dinger picture, the state would evolve as time goes on.
As mentioned, the scattering state would be continuous \textit{superposition}\footnote{It would be discrete sum for a bound state.} of states with a spread of energies,
\begin{equation}
    \ket*{\Psi_{g}(t)} = \int{d^{3}k/(2\pi)^3\,g(\mathbf{k})\exp(-i\frac{\hbar t\mathbf{k}^2}{2\mu})\ket*{\Psi_{\mathbf{k}}^{\text{in}}}}
\end{equation}
where $g(\mathbf{k})$ is a smooth function that is peaked around some wave number $\mathbf{k}_0$.
The eigenstate $\ket*{\Psi_{\mathbf{k}}^{\text{in}}}$ satisfies the further condition that for any sufficiently smooth  function $g(\mathbf{k})$, in the limit $t\rightarrow-\infty$,
\begin{equation}
    \ket*{\Psi_{g}(t)}\rightarrow\int{d^3k/(2\pi)^3\,g(\mathbf{k})\exp(-i\frac{\hbar t\mathbf{k}^2}{2\mu})\ket*{\Phi_{\mathbf{k}}}}
\end{equation}
where $\ket*{\Phi_{\mathbf{k}}}$ are orthonormal eigenstates of the momentum operator $\mathbf{P}$ with eigenvalue $\hbar\mathbf{k}$,
\begin{equation}
    \begin{split}
        \mathbf{P}\ket*{\Phi_{\mathbf{k}}} = \hbar\mathbf{k}\ket*{\Phi_{\mathbf{k}}}\\
        \braket*{\Phi_{\mathbf{k}}}{\Phi_{\mathbf{k}'}} = (2\pi)^3\delta^{(3)}(\hbar\mathbf{k} - \hbar\mathbf{k}')
    \end{split}
\end{equation}
and hence eigenstates of $H_0$ (not H), with eigenvalue $E(|\mathbf{k}|) = \hbar^2\mathbf{k}^2/2\mu$. 
Note that $\ket*{\Psi_{\mathbf{k}}^{\text{in}}}\text{ and } \ket*{\Phi_{\mathbf{k}}}$ belong to different Hilbert spaces.
The incident wave packet $\ket*{\Psi_{g}(t)}$ should satisfy the normalization condition $\braket*{\Psi_{g}(t)}{\Psi_{g}(t)} = 1$, which is equivalent to the condition at time limit,
\[\int\int{d^3kd^{3}k'/(2\pi)^3\,g^{*}(\mathbf{k})g(\mathbf{k}')\exp(-i\frac{\hbar t}{2\mu}(\mathbf{k}'^2-\mathbf{k}^2))\delta^{(3)}(\hbar\mathbf{k} - \hbar\mathbf{k}')}\]
\begin{equation}
    = \hbar^{-3}\int{d^3k/(2\pi)^3\,|g(\mathbf{k})|^2} = 1
\end{equation}

Rewrite the equation (1.2) as,
\begin{equation}
    \left(E(|\mathbf{k}|)-H_0\right)\ket*{\Psi_{\mathbf{k}}^{\text{in}}} = V\ket*{\Psi_{\mathbf{k}}^{\text{in}}}
\end{equation}
This has a formal solution,
\begin{equation}
    \ket*{\Psi_{\mathbf{k}}^{\text{in}}} = \ket*{\Phi_{\mathbf{k}}} + \left(E(|\mathbf{k}|)-H_0+i\epsilon\right)^{-1}\ket*{\Psi_{\mathbf{k}}^{\text{in}}}
\end{equation}
The first term on RHS could always be added up because it is the \textit{homogeneous} solution of equation (1.7), and $\epsilon$ is a positive infinitesimal quantity, which is inserted to give meaning to the operator $(E(|\mathbf{k}|)-H_0+i\epsilon)^{-1}$ when we integrate over the eigenvalues of $H_0$\footnote{It is more reasonable in path integral formalism that the additional term must be added to prevent the integral diverges at large value of wave function.}.
It is known as the \textit{Lippmann-Schwinger equation}.
The special feature of the particular 'solution' is that it also satisfies the additional initial condition.
\\\indent To see this, we could expand $V\ket*{\Psi_{\mathbf{k}}^{\text{in}}}$ in the orthonormal free-particle states $\ket*{\Phi_{\mathbf{q}}}$:
\begin{equation}
    V\ket*{\Psi_{\mathbf{k}}^{\text{in}}} = \int{d^3q/(2\pi)^3\,\ket*{\Phi_{\mathbf{q}}}\bra*{\Phi_{\mathbf{q}}}V\ket*{\Psi_{\mathbf{k}}^{\text{in}}}}
\end{equation}
Thus equation (1.8) becomes,
\[\ket*{\Psi_{\mathbf{k}}^{\text{in}}} = \ket*{\Phi_{\mathbf{k}}} + \hbar^3\int{d^3q/(2\pi)^3\,\left(E(|\mathbf{k}|) - H_0 + i\epsilon\right)^{-1}\ket*{\Phi_{\mathbf{q}}}\bra*{\Phi_{\mathbf{q}}}V\ket*{\Psi_{\mathbf{k}}^{\text{in}}}}\]
\begin{equation}
    = \ket*{\Phi_{\mathbf{k}}} + \hbar^3\int{d^3q/(2\pi)^3\,\left(E(|\mathbf{k}|) - E(|\mathbf{q}|) + i\epsilon\right)^{-1}\ket*{\Phi_{\mathbf{q}}}\bra*{\Phi_{\mathbf{q}}}V\ket*{\Psi_{\mathbf{k}}^{\text{in}}}}
\end{equation}
In calculating the integral over $\mathbf{k}$ in equation (1.3),
\[\int{d^3k/(2\pi)^3\,g(\mathbf{k})\frac{\exp(-i\hbar t\mathbf{k}^2/2\mu)}{E(|\mathbf{k}|)-E(q)+i\epsilon}\bra*{\Phi_{\mathbf{q}}}V\ket*{\Psi_{\mathbf{k}}^{\text{in}}}}\]
\[=\int{d\Omega}\int_{0}^{\infty}{dk/(2\pi)^3\,k^2g(\mathbf{k})}\frac{\exp(-i\hbar t\mathbf{k}^2/2\mu)}{E(k)-E(q)+i\epsilon}\bra*{\Phi_{\mathbf{q}}}V\ket*{\Psi_{\mathbf{k}}^{\text{in}}}\]
where $d\Omega = \sin\theta\,d\theta d\phi$. We could convert the integral over $k$ to an integral over the kinetic energy, using $dk = \mu dE/k\hbar^2$. Now, when $t\rightarrow-\infty$, the exponential term oscillates very rapidly, so that the only value of $E$ that contribute are \textit{those very near} $E(q)$, where the denominator also varies very rapidly.
\[=\int{d\Omega}/(2\pi)^3\int_{0}^{\infty}{dE\,\frac{\mu k}{\hbar^2}g(\mathbf{k})}\frac{\exp(-iEt/\hbar)}{E-E(q)+i\epsilon}\bra*{\Phi_{\mathbf{q}}}V\ket*{\Psi_{\mathbf{k}}^{\text{in}}}\]
Hence, for the time limit $t\rightarrow-\infty$, we could set $k = q$ everywhere except in the rapidly varying exponential and denominator,
\[=\int{d\Omega/(2\pi)^3\,g(\Omega)\frac{\mu q}{\hbar^2}}\bra*{\Phi_{\mathbf{q}}}V\ket*{\Psi_{\mathbf{q}}^{\text{in}}}\int_{\text{around E(q)}}{dE\,\frac{\exp(-iEt/\hbar)}{E-E(q)+i\epsilon}}\]
We could extend the integration range to the whole real axis, which is permissible because the integral receives no appreciable contributions anywhere that is far from $E(q)$,
\begin{equation}
    \varpropto \int_{-\infty}^{\infty}{dE\,\frac{\exp(-iEt/\hbar)}{E-E(q)+i\epsilon}}
\end{equation}
For $t\rightarrow-\infty$, we can close the contour of the integration with a very large semi-circle in the upper half of the complex plane, on which the integration is negligible because, for $\Im(E)>0$ and $t\rightarrow-\infty$, the numerator $\exp(-iEt/\hbar)$ is exponentially small. 
The only singularity of the integration is a pole at $E = E(q)-i\epsilon$, which is always in the lower half plane because $\epsilon$ is positive infinitesimal parameter.
Hence, the integral vanishes for $t\rightarrow-\infty$.
Only the first term of equation (1.10) is left, which gives the correct condition for $t\rightarrow-\infty$
\subsection{Scattering Amplitudes}
In the previous section, we defined a state that at early times has the appearance of a particle traveling toward a collision with a scattering center. 
Now we must consider what this state looks like after the collision.
\\\indent For this purpose, we consider the coordinate-space wave function of the state $\ket*{\Psi_{\mathbf{k}}^{\text{in}}}$\footnote{By this way, it would give us a physically vivid picture about scattering.}.
\begin{equation}
    V\ket*{\Psi_{\mathbf{k}}^{\text{in}}} = \int{d^3x\,\ket*{\Phi_{\mathbf{x}}}\bra*{\Phi_{\mathbf{x}}}V\ket*{\Psi_{\mathbf{k}}^{\text{in}}}} = \int{d^3x\,\ket*{\Phi_{\mathbf{x}}}V(\mathbf{x})\psi_{\mathbf{k}}(\mathbf{x})}
\end{equation}
where $\psi_{\mathbf{k}}(\mathbf{x})$ is the coordinate-space wave function of the in-state,
\begin{equation}
    \psi_{\mathbf{k}}(\mathbf{x}) = \braket*{\Phi_{\mathbf{x}}}{\Psi_{\mathbf{k}}^{\text{in}}}
\end{equation}
Then, by taking the scalar product of the Lippmann-Schwinger equation (1.8), and using the fact that the scalar product of state of definite momentum and state of definite position would be plane-wave function,
\[\braket*{\Phi_{\mathbf{x}}}{\Phi_{\mathbf{k}}} = e^{i\mathbf{k}\cdot\mathbf{x}}\]
we have,
\[\psi_{\mathbf{k}}(\mathbf{x}) = e^{i\mathbf{k}\cdot\mathbf{x}} + \bra*{\Phi_{\mathbf{x}}}\left(E(|\mathbf{k}|)-H_0+i\epsilon\right)^{-1}V\ket*{\Psi_{\mathbf{k}}^{\text{in}}}\]
\begin{equation}
    \psi_{\mathbf{k}}(\mathbf{x}) = e^{i\mathbf{k}\cdot\mathbf{x}} + \int{d^3y\,\bra*{\Phi_{\mathbf{x}}}\left[E(|\mathbf{k}|)-H_0+i\epsilon\right]^{-1}\ket*{\Phi_{\mathbf{y}}}V(y)\psi_{\mathbf{k}}(\mathbf{y})}
\end{equation}
where we define the Green function and then evaluate it by Fourier transform,
\[
    G_{k}(\mathbf{x}-\mathbf{y}) = \bra*{\Phi_{\mathbf{x}}}\left[E(|\mathbf{k}|)-H_0+i\epsilon\right]^{-1}\ket*{\Phi_{\mathbf{y}}}
\]
\[ = \int{d^3q/(2\pi)^3\,\frac{e^{i\mathbf{q}\cdot(\mathbf{x}-\mathbf{y})}}{E(|\mathbf{k}|)-E(|\mathbf{q}|)+i\epsilon}}\]
\[=\int{-d\cos(\theta) d\phi}\int{q^2dq/(2\pi)^3\,\frac{e^{iq|\mathbf{x}-\mathbf{y}|\cos(\theta)}}{\hbar^2k^2/2\mu-\hbar^2q^2/2\mu+i\epsilon}}\]
\[ = \frac{2\pi}{(2\pi)^3}\int_{0}^{\infty}{q^2dq\,\frac{2i\sin(q|\mathbf{x}-\mathbf{y}|)}{iq|\mathbf{x}-\mathbf{y}|}\frac{2\mu/\hbar^2}{k^2-q^2+i\epsilon}}\]
\[=-i\frac{2\mu}{\hbar^2}\frac{1}{4\pi^2|\mathbf{x}-\mathbf{y}|}\int_{-\infty}^{\infty}{dq\,\frac{qe^{iq|\mathbf{x}-\mathbf{y}|}}{k^2-q^2+i\epsilon}}\]
\begin{equation}
    =-\frac{2\mu}{\hbar^2}\frac{1}{4\pi|\mathbf{x}-\mathbf{y}|}e^{ik|\mathbf{x}-\mathbf{y}|}
\end{equation}
The last line could be evaluated by completing the contour integral with a large semi-circle in the upper half plane.
The poles are $q = \pm\sqrt{k^2-i\epsilon} = k\pm i\epsilon$, and picking up the contribution of pole $q = k+i\epsilon$.
\[2\pi i\lim_{\epsilon\rightarrow0}\frac{(k+i\epsilon)e^{i(k+i\epsilon)|\mathbf{x}-\mathbf{y}|}}{-(k+q+i\epsilon)} = -\pi ie^{ik|\mathbf{x}-\mathbf{y}|}\]
For a potential $V(\mathbf{y})$ that vanishes sufficiently rapidly as $|\mathbf{y}|\rightarrow\infty$.
Write equation (1.14)
\begin{equation}
    \psi_{\mathbf{k}}(\mathbf{x}) = e^{i\mathbf{k}\cdot\mathbf{x}} + \int{d^3y\,G_{\mathbf{k}}(\mathbf{x}-\mathbf{y})V(\mathbf{y})\psi_{\mathbf{k}}(\mathbf{y})}
\end{equation}
For $|\mathbf{x}|\rightarrow\infty$,
\begin{equation}
    \psi_{\mathbf{k}}(\mathbf{x})\rightarrow e^{i\mathbf{k}\cdot\mathbf{x}}+f_{\mathbf{k}}(\hat{x})e^{ikr}/r
\end{equation}
where $|\mathbf{x}|\equiv r$ and $f_{\mathbf{k}}(\hat{x})$ is \textit{scattering amplitude}.
\begin{equation}
    f_{\mathbf{k}}(\hat{x}) = -\frac{\mu}{2\pi\hbar^2}\int{d^3y\,e^{-ik\hat{x}\cdot\mathbf{y}}V(\mathbf{y})\psi_{\mathbf{k}}(\mathbf{y})}
\end{equation}
There is an interesting result that the coordinate-space wave function would have incident-coming wave part and scattering wave part, and the scattering wave part would act like spherical wave.
\\\indent Now let's consider how the superposition equation (1.3),
\begin{equation}
    \psi_{g}(\mathbf{x}, t)\equiv\braket*{\Phi_{\mathbf{x}}}{\Psi_{g}(t)} = \int{d^3k\,g(\mathbf{k})\psi_{\mathbf{k}}(\mathbf{x})\exp(-i\hbar t\mathbf{k}^2/2\mu)}
\end{equation}
\end{document}