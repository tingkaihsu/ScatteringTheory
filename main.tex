\documentclass[12pt]{article}
\usepackage{braket}
\usepackage{physics}
\usepackage{graphicx}
\usepackage{times}
\usepackage[export]{adjustbox}
\usepackage{listings}
\usepackage{mathcomp}
\usepackage{hyperref}
\usepackage{bm,amsmath}
\usepackage{amssymb}
\usepackage{float}
\usepackage{indentfirst}
\usepackage{bigints}
\usepackage{listings}
\usepackage{color}
\hypersetup{
colorlinks=true,
linkcolor=blue,
filecolor=magenta,
urlcolor=cyan,
pdftitle={Overleaf Example},
pdfpagemode=FullScreen,
}
\definecolor{dkgreen}{rgb}{0,0.6,0}
\definecolor{gray}{rgb}{0.5,0.5,0.5}
\definecolor{mauve}{rgb}{0.58,0,0.82}
\lstset{frame=tb,
language=Python,
aboveskip=3mm,
belowskip=3mm,
stepnumber = 1,
showstringspaces=false,
columns=flexible,
basicstyle={\small\ttfamily},
numbers=left,
numberstyle=\color{gray},
keywordstyle=\color{blue},
commentstyle=\color{dkgreen},
stringstyle=\color{mauve},
breaklines=true,
breakatwhitespace=true,
tabsize=3
}
\usepackage{xcolor}
\usepackage{lipsum}
\newcommand{\cstars}{\vskip 1.2ex\noindent\par\makebox[\linewidth]{\color{cyan}* * * }\par}
\numberwithin{equation}{section}


\title{Scattering Theory}
\author{Ting-Kai Hsu}
\date{\today}

\begin{document}
\maketitle
\tableofcontents
\part{Quantum Mechanics}
This article is designed for leading the reader from old scattering theory of quantum mechanics to nowadays scattering theory of quantum field theory.
The quantum mechanics part would be based on \textit{Lectures on Quantum Mechanics} by S.Weinberg.
\section{Potential Scattering}
In this chapter we would study the theory of scattering in a simple but important case, the elastic scattering of a non-relativistic particle in a local potential, but using modern techniques that could easily be extended to more general problems.
\subsection{In-States}
Consider a non-relativistic particle of mass $\mu$ in a potential $V(\mathbf{x})$, whose Hamiltonian is,
\begin{equation}
    H = H_0 + V(\mathbf{x})
\end{equation}
Here $H_0 = \mathbf{p}^2/2\mu$ is the kinetic energy operator. 
The potential is assumed to be function of position operator, and tends to zero as $|\mathbf{x}| = r\rightarrow\infty$.
Then it is concerned here with a positive-energy particle\footnote{Actually, there could possibly be negative energy if it is in bound state, that is, the energy is less than potential at boundaries. But now, we are considering scattering state, which is opposite with bound state.}.
The particle comes into facing the potential from great distances regarded to have no interaction with momentum $\hbar\mathbf{k}$, and is scattered, going out again to infinity, generally along a different direction.
\\\indent In Heisenberg picture, for a particle with momentum $\hbar\mathbf{k}$ far from the scattering center if the measurement of the particle is made at very early times, and this situation is represented by a time-independent state vector $\ket*{\Psi^{\text{in}}_{\mathbf{k}}}$.
As mentioned, at very early times the particle is at a location that is far from the scattering center so that the potential is negligible there, so the state has an energy $\hbar^2\mathbf{k}^2/2\mu$, with the following relation,
\begin{equation}
    H\ket*{\Psi_{\mathbf{k}}^{\text{in}}} = \frac{\hbar^2\mathbf{k}^2}{2\mu}\ket*{\Psi_{\mathbf{k}}^{\text{in}}}
\end{equation}

Now switch to Schr$\ddot{\text{o}}$dinger picture, the state would evolve as time goes on.
As mentioned, the scattering state would be continuous \textit{superposition}\footnote{It would be discrete sum for a bound state.} of states with a spread of energies,
\begin{equation}
    \ket*{\Psi_{g}(t)} = \int{d^{3}k/(2\pi)^3\,g(\mathbf{k})\exp(-i\frac{\hbar t\mathbf{k}^2}{2\mu})\ket*{\Psi_{\mathbf{k}}^{\text{in}}}}
\end{equation}
where $g(\mathbf{k})$ is a smooth function that is peaked around some wave number $\mathbf{k}_0$.
The eigenstate $\ket*{\Psi_{\mathbf{k}}^{\text{in}}}$ satisfies the further condition that for any sufficiently smooth  function $g(\mathbf{k})$, in the limit $t\rightarrow-\infty$,
\begin{equation}
    \ket*{\Psi_{g}(t)}\rightarrow\int{d^3k/(2\pi)^3\,g(\mathbf{k})\exp(-i\frac{\hbar t\mathbf{k}^2}{2\mu})\ket*{\Phi_{\mathbf{k}}}}
\end{equation}
where $\ket*{\Phi_{\mathbf{k}}}$ are orthonormal eigenstates of the momentum operator $\mathbf{P}$ with eigenvalue $\hbar\mathbf{k}$,
\begin{equation}
    \begin{split}
        \mathbf{P}\ket*{\Phi_{\mathbf{k}}} = \hbar\mathbf{k}\ket*{\Phi_{\mathbf{k}}}\\
        \braket*{\Phi_{\mathbf{k}}}{\Phi_{\mathbf{k}'}} = (2\pi)^3\delta^{(3)}(\hbar\mathbf{k} - \hbar\mathbf{k}')
    \end{split}
\end{equation}
and hence eigenstates of $H_0$ (not H), with eigenvalue $E(|\mathbf{k}|) = \hbar^2\mathbf{k}^2/2\mu$. 
Note that $\ket*{\Psi_{\mathbf{k}}^{\text{in}}}\text{ and } \ket*{\Phi_{\mathbf{k}}}$ belong to different Hilbert spaces.
The incident wave packet $\ket*{\Psi_{g}(t)}$ should satisfy the normalization condition $\braket*{\Psi_{g}(t)}{\Psi_{g}(t)} = 1$, which is equivalent to the condition at time limit,
\[\int\int{d^3kd^{3}k'/(2\pi)^3\,g^{*}(\mathbf{k})g(\mathbf{k}')\exp(-i\frac{\hbar t}{2\mu}(\mathbf{k}'^2-\mathbf{k}^2))\delta^{(3)}(\hbar\mathbf{k} - \hbar\mathbf{k}')}\]
\begin{equation}
    = \hbar^{-3}\int{d^3k/(2\pi)^3\,|g(\mathbf{k})|^2} = 1
\end{equation}

Rewrite the equation (1.2) as,
\begin{equation}
    \left(E(|\mathbf{k}|)-H_0\right)\ket*{\Psi_{\mathbf{k}}^{\text{in}}} = V\ket*{\Psi_{\mathbf{k}}^{\text{in}}}
\end{equation}
This has a formal solution,
\begin{equation}
    \ket*{\Psi_{\mathbf{k}}^{\text{in}}} = \ket*{\Phi_{\mathbf{k}}} + \left(E(|\mathbf{k}|)-H_0+i\epsilon\right)^{-1}\ket*{\Psi_{\mathbf{k}}^{\text{in}}}
\end{equation}
The first term on RHS could always be added up because it is the \textit{homogeneous} solution of equation (1.7), and $\epsilon$ is a positive infinitesimal quantity, which is inserted to give meaning to the operator $(E(|\mathbf{k}|)-H_0+i\epsilon)^{-1}$ when we integrate over the eigenvalues of $H_0$\footnote{It is more reasonable in path integral formalism that the additional term must be added to prevent the integral diverges at large value of wave function.}.
It is known as the \textit{Lippmann-Schwinger equation}.
The special feature of the particular 'solution' is that it also satisfies the additional initial condition.
\\\indent To see this, we could expand $V\ket*{\Psi_{\mathbf{k}}^{\text{in}}}$ in the orthonormal free-particle states $\ket*{\Phi_{\mathbf{q}}}$:
\begin{equation}
    V\ket*{\Psi_{\mathbf{k}}^{\text{in}}} = \int{d^3q/(2\pi)^3\,\ket*{\Phi_{\mathbf{q}}}\bra*{\Phi_{\mathbf{q}}}V\ket*{\Psi_{\mathbf{k}}^{\text{in}}}}
\end{equation}
Thus equation (1.8) becomes,
\[\ket*{\Psi_{\mathbf{k}}^{\text{in}}} = \ket*{\Phi_{\mathbf{k}}} + \hbar^3\int{d^3q/(2\pi)^3\,\left(E(|\mathbf{k}|) - H_0 + i\epsilon\right)^{-1}\ket*{\Phi_{\mathbf{q}}}\bra*{\Phi_{\mathbf{q}}}V\ket*{\Psi_{\mathbf{k}}^{\text{in}}}}\]
\begin{equation}
    = \ket*{\Phi_{\mathbf{k}}} + \hbar^3\int{d^3q/(2\pi)^3\,\left(E(|\mathbf{k}|) - E(|\mathbf{q}|) + i\epsilon\right)^{-1}\ket*{\Phi_{\mathbf{q}}}\bra*{\Phi_{\mathbf{q}}}V\ket*{\Psi_{\mathbf{k}}^{\text{in}}}}
\end{equation}
In calculating the integral over $\mathbf{k}$ in equation (1.3),
\[\int{d^3k/(2\pi)^3\,g(\mathbf{k})\frac{\exp(-i\hbar t\mathbf{k}^2/2\mu)}{E(|\mathbf{k}|)-E(q)+i\epsilon}\bra*{\Phi_{\mathbf{q}}}V\ket*{\Psi_{\mathbf{k}}^{\text{in}}}}\]
\[=\int{d\Omega}\int_{0}^{\infty}{dk/(2\pi)^3\,k^2g(\mathbf{k})}\frac{\exp(-i\hbar t\mathbf{k}^2/2\mu)}{E(k)-E(q)+i\epsilon}\bra*{\Phi_{\mathbf{q}}}V\ket*{\Psi_{\mathbf{k}}^{\text{in}}}\]
where $d\Omega = \sin\theta\,d\theta d\phi$. We could convert the integral over $k$ to an integral over the kinetic energy, using $dk = \mu dE/k\hbar^2$. Now, when $t\rightarrow-\infty$, the exponential term oscillates very rapidly, so that the only value of $E$ that contribute are \textit{those very near} $E(q)$, where the denominator also varies very rapidly.
\[=\int{d\Omega}/(2\pi)^3\int_{0}^{\infty}{dE\,\frac{\mu k}{\hbar^2}g(\mathbf{k})}\frac{\exp(-iEt/\hbar)}{E-E(q)+i\epsilon}\bra*{\Phi_{\mathbf{q}}}V\ket*{\Psi_{\mathbf{k}}^{\text{in}}}\]
Hence, for the time limit $t\rightarrow-\infty$, we could set $k = q$ everywhere except in the rapidly varying exponential and denominator,
\[=\int{d\Omega/(2\pi)^3\,g(\Omega)\frac{\mu q}{\hbar^2}}\bra*{\Phi_{\mathbf{q}}}V\ket*{\Psi_{\mathbf{q}}^{\text{in}}}\int_{\text{around E(q)}}{dE\,\frac{\exp(-iEt/\hbar)}{E-E(q)+i\epsilon}}\]
We could extend the integration range to the whole real axis, which is permissible because the integral receives no appreciable contributions anywhere that is far from $E(q)$,
\begin{equation}
    \varpropto \int_{-\infty}^{\infty}{dE\,\frac{\exp(-iEt/\hbar)}{E-E(q)+i\epsilon}}
\end{equation}
For $t\rightarrow-\infty$, we can close the contour of the integration with a very large semi-circle in the upper half of the complex plane, on which the integration is negligible because, for $\Im(E)>0$ and $t\rightarrow-\infty$, the numerator $\exp(-iEt/\hbar)$ is exponentially small. 
The only singularity of the integration is a pole at $E = E(q)-i\epsilon$, which is always in the lower half plane because $\epsilon$ is positive infinitesimal parameter.
Hence, the integral vanishes for $t\rightarrow-\infty$.
Only the first term of equation (1.10) is left, which gives the correct condition for $t\rightarrow-\infty$
\subsection{Scattering Amplitudes}
In the previous section, we defined a state that at early times has the appearance of a particle traveling toward a collision with a scattering center. 
Now we must consider what this state looks like after the collision.
\\\indent For this purpose, we consider the coordinate-space wave function of the state $\ket*{\Psi_{\mathbf{k}}^{\text{in}}}$\footnote{By this way, it would give us a physically vivid picture about scattering.}.
\begin{equation}
    V\ket*{\Psi_{\mathbf{k}}^{\text{in}}} = \int{d^3x\,\ket*{\Phi_{\mathbf{x}}}\bra*{\Phi_{\mathbf{x}}}V\ket*{\Psi_{\mathbf{k}}^{\text{in}}}} = \int{d^3x\,\ket*{\Phi_{\mathbf{x}}}V(\mathbf{x})\psi_{\mathbf{k}}(\mathbf{x})}
\end{equation}
where $\psi_{\mathbf{k}}(\mathbf{x})$ is the coordinate-space wave function of the in-state,
\begin{equation}
    \psi_{\mathbf{k}}(\mathbf{x}) = \braket*{\Phi_{\mathbf{x}}}{\Psi_{\mathbf{k}}^{\text{in}}}
\end{equation}
Then, by taking the scalar product of the Lippmann-Schwinger equation (1.8), and using the fact that the scalar product of state of definite momentum and state of definite position would be plane-wave function,
\[\braket*{\Phi_{\mathbf{x}}}{\Phi_{\mathbf{k}}} = e^{i\mathbf{k}\cdot\mathbf{x}}\]
we have,
\[\psi_{\mathbf{k}}(\mathbf{x}) = e^{i\mathbf{k}\cdot\mathbf{x}} + \bra*{\Phi_{\mathbf{x}}}\left(E(|\mathbf{k}|)-H_0+i\epsilon\right)^{-1}V\ket*{\Psi_{\mathbf{k}}^{\text{in}}}\]
\begin{equation}
    \psi_{\mathbf{k}}(\mathbf{x}) = e^{i\mathbf{k}\cdot\mathbf{x}} + \int{d^3y\,\bra*{\Phi_{\mathbf{x}}}\left[E(|\mathbf{k}|)-H_0+i\epsilon\right]^{-1}\ket*{\Phi_{\mathbf{y}}}V(y)\psi_{\mathbf{k}}(\mathbf{y})}
\end{equation}
where we define the Green function and then evaluate it by Fourier transform,
\[
    G_{k}(\mathbf{x}-\mathbf{y}) = \bra*{\Phi_{\mathbf{x}}}\left[E(|\mathbf{k}|)-H_0+i\epsilon\right]^{-1}\ket*{\Phi_{\mathbf{y}}}
\]
\[ = \int{d^3q/(2\pi)^3\,\frac{e^{i\mathbf{q}\cdot(\mathbf{x}-\mathbf{y})}}{E(|\mathbf{k}|)-E(|\mathbf{q}|)+i\epsilon}}\]
\[=\int{-d\cos(\theta) d\phi}\int{q^2dq/(2\pi)^3\,\frac{e^{iq|\mathbf{x}-\mathbf{y}|\cos(\theta)}}{\hbar^2k^2/2\mu-\hbar^2q^2/2\mu+i\epsilon}}\]
\[ = \frac{2\pi}{(2\pi)^3}\int_{0}^{\infty}{q^2dq\,\frac{2i\sin(q|\mathbf{x}-\mathbf{y}|)}{iq|\mathbf{x}-\mathbf{y}|}\frac{2\mu/\hbar^2}{k^2-q^2+i\epsilon}}\]
\[=-i\frac{2\mu}{\hbar^2}\frac{1}{4\pi^2|\mathbf{x}-\mathbf{y}|}\int_{-\infty}^{\infty}{dq\,\frac{qe^{iq|\mathbf{x}-\mathbf{y}|}}{k^2-q^2+i\epsilon}}\]
\begin{equation}
    =-\frac{2\mu}{\hbar^2}\frac{1}{4\pi|\mathbf{x}-\mathbf{y}|}e^{ik|\mathbf{x}-\mathbf{y}|}
\end{equation}
The last line could be evaluated by completing the contour integral with a large semi-circle in the upper half plane.
The poles are $q = \pm\sqrt{k^2-i\epsilon} = k\pm i\epsilon$, and picking up the contribution of pole $q = k+i\epsilon$.
\[2\pi i\lim_{\epsilon\rightarrow0}\frac{(k+i\epsilon)e^{i(k+i\epsilon)|\mathbf{x}-\mathbf{y}|}}{-(k+q+i\epsilon)} = -\pi ie^{ik|\mathbf{x}-\mathbf{y}|}\]
For a potential $V(\mathbf{y})$ that vanishes sufficiently rapidly as $|\mathbf{y}|\rightarrow\infty$.
Write equation (1.14)
\begin{equation}
    \psi_{\mathbf{k}}(\mathbf{x}) = e^{i\mathbf{k}\cdot\mathbf{x}} + \int{d^3y\,G_{\mathbf{k}}(\mathbf{x}-\mathbf{y})V(\mathbf{y})\psi_{\mathbf{k}}(\mathbf{y})}
\end{equation}
For $|\mathbf{x}|\rightarrow\infty$,
\begin{equation}
    \psi_{\mathbf{k}}(\mathbf{x})\rightarrow e^{i\mathbf{k}\cdot\mathbf{x}}+f_{\mathbf{k}}(\hat{x})e^{ikr}/r
\end{equation}
where $|\mathbf{x}|\equiv r$ and $f_{\mathbf{k}}(\hat{x})$ is \textit{scattering amplitude}.
\begin{equation}
    f_{\mathbf{k}}(\hat{x}) = -\frac{\mu}{2\pi\hbar^2}\int{d^3y\,e^{-ik\hat{x}\cdot\mathbf{y}}V(\mathbf{y})\psi_{\mathbf{k}}(\mathbf{y})}
\end{equation}
There is an interesting result that the coordinate-space wave function would have incident-coming wave part and scattering wave part, and the scattering wave part would act like spherical wave.
\\\indent Now let's consider how the superposition equation (1.3),
\begin{equation}
    \psi_{g}(\mathbf{x}, t)\equiv\braket*{\Phi_{\mathbf{x}}}{\Psi_{g}(t)} = \int{d^3k/(2\pi)^3\,g(\mathbf{k})\psi_{\mathbf{k}}(\mathbf{x})\exp(-i\hbar t\mathbf{k}^2/2\mu)}
\end{equation}
in the limit $t\rightarrow+\infty$, with $r/t$ held fixed, and $\mathbf{x}$ off the 3-axis\footnote{Which means the first term in equation (1.17) vanishes.}, equation (1.19) gives\footnote{We would assume that the particle comes in from a great distance along the negative 3-axis, so we are interested in the limit of very large negative $t$ and $x_3$, but with $x_3/t$ held finite. And also assume the particle velocity is sufficiently closely confined to the 3-direction, where the function $g(\mathbf{k})$ is not negligible.},
\[\psi_{g}(\mathbf{x}, t)\rightarrow\frac{1}{(2\pi)^3r}\int{d^2k_{\perp}}\int_{-\infty}^{\infty}{dk_3\,g(\mathbf{k}_{\perp}, k_3)}\]
\begin{equation}
    \times\exp(ik_3r-i\hbar tk_3^2/2\mu)f_{\mathbf{k}_0}(\hat{x})
\end{equation}
We have taken the subscript on the scattering amplitude to be $\mathbf{k}_{0}$, because the function $g$ is sharply peaked at this value of $\mathbf{k}$, and we have approximated $k = \sqrt{k_3^2+\mathbf{k}_{\perp}^2}$ as $k\approx k_3$ in the exponent, because $g(\mathbf{k}_{\perp}, k_3)$ is assumed to be negligible except for $|\mathbf{k}_{\perp}|<<k_3$.
That is, we assume that the function $g(\mathbf{k}_{\perp}, k_3)$, though smooth, is strongly peaked at $k_3=k_0$ and $\mathbf{k}_{\perp} = 0$, so we could set $k_3$ in $g(\mathbf{k}_{\perp}, k_3)$ equal to the value $\mu r/\hbar t$, so that,
\[
    \psi_{g}(\mathbf{x}, t)\rightarrow\frac{1}{(2\pi)^3r}f_{\mathbf{k}_0}(\hat{x})\int{d^2k_{\perp}\,g(\mathbf{k}_{\perp}, \mu r/\hbar t)}
\]
\[\times\int_{-\infty}^{\infty}{dk_3\,\exp(ik_3r-i\hbar tk_3^2/2\mu)}\]
\begin{equation}
    =\frac{1}{(2\pi)^3r}f_{\mathbf{k}_0}\int{d^2k_{\perp}\,g(\mathbf{k}_{\perp}, \mu r/\hbar t)\exp(i\mu r^2/2\hbar t)\sqrt{\frac{2\pi\,\mu}{i\hbar t}}}
\end{equation}
Where we reach the last line by Gaussian integral.
\\\indent The probability $dP(\hat{x})$ that the particle at late times is somewhere within the cone of infinitesimal solid angle $d\omega$ around the direction $\hat{x}$ is then the integral of $|\psi_{g}(\mathbf{x}, t)|^2$ over this cone:
\begin{equation}
    dP(\hat{x}, \mathbf{k}_0) = d\Omega\int_{0}^{\infty}{r^2dr\,|\psi_{g}(\mathbf{x}, t)|^2}
\end{equation}
\[\rightarrow d\Omega\frac{2\mu\pi}{(2\pi)^6\hbar t}|f_{\mathbf{k}_0}(\hat{x})|^2\int_{0}^{\infty}{dr\,\left|\int{d^2k_{\perp}\,g(\mathbf{k}_{\perp}, \mu r/\hbar t)}\right|^2}\]
changing the variable of integration $r$ to $k_3\equiv\mu r/\hbar t$,
\begin{equation}
    \frac{d\,P(\hat{x}, \mathbf{k}_0)}{d\Omega} = |f_{\mathbf{k}_0}(\hat{x})|^2\int_{0}^{\infty}{dk_3/(2\pi)^5\,\left|\int{d^2k_{\perp}\,g(\mathbf{k}_{\perp}, k_3)}\right|^2}
\end{equation}

Now, the coefficient of $|f_{\mathbf{k}_0}(\hat{x})|^2$ in equation (1.23) has the dimensions of an inverse area. In fact, it is precisely the probability per unit area that the particle is in a small area centered on the 3-axis and normal to that axis:
\begin{equation}
    \rho_{\perp}\equiv\lim_{t\rightarrow-\infty}\int_{-\infty}^{\infty}{dx_{3}\,\left|\psi_{g}(0, x_3, t)\right|^2}
\end{equation}
To see this, recall equation (1.4) and its scalar product with state of definite position,
\[\braket*{\Phi_{\mathbf{x}}}{\Psi_{g}(t)}\rightarrow\int{d^3k/(2\pi)^3\,g(\mathbf{k})\exp(i\mathbf{k}\cdot\mathbf{x}-i\hbar t\mathbf{k}^2/2\mu)}\]
As what we've done in equation (1.21), we assume that the particle moves along the 3-axis,
\[\braket*{\Phi_{\mathbf{x}}}{\Psi_{g}(t)}\rightarrow\]\[\frac{1}{(2\pi)^3}\int{d^2k_{\perp}}\int{dk_3\,g(\mathbf{k}_{\perp}, k_3)\exp\left(i\mathbf{k}_{\perp}\cdot\mathbf{x}_{\perp}+ik_3x_3-i\hbar t(k_3-\mu x_3/\hbar t)^2/2\mu\right)}\]
Replace $k_3$ with $k_3 = \mu x_3/\hbar t$ except the exponent term.
\[\braket*{\Phi_{\mathbf{x}}}{\Psi_{g}(t)}\rightarrow\]
\[\frac{1}{(2\pi)^3}\int{d^2k_{\perp}\,g(\mathbf{k}_{\perp}, \mu x_3/\hbar t)}\exp(i\mathbf{k}_{\perp}\cdot\mathbf{x}_{\perp})\]
\[\times\exp(ix^2_3\mu/2\hbar t)\int_{-\infty}^{\infty}{dk_3\,\exp\left(-i\hbar t (k_3-\mu x_3/\hbar t)^2/2\mu\right)}\]
\begin{equation}
    =\frac{1}{(2\pi)^3}\exp(ix^2_3\mu/2\hbar t)\sqrt{\frac{2\mu\pi}{i\hbar t}}\times\int{d^2k_{\perp}\,g(\mathbf{k}_{\perp},\mu x_3/\hbar t)\exp(i\mathbf{k}_{\perp}\cdot\mathbf{x}_{\perp})}
\end{equation}
Therefore, in particular as $t$ tends to infinity far ($\infty$), the spatial probability distribution is,
\begin{equation}
    \left|\braket*{\Phi_{\mathbf{x}}}{\Psi_{g}(t)}\right|^2\rightarrow\frac{\mu}{\hbar t(2\pi)^5}\left|\int{d^2k_{\perp}\,g(\mathbf{k}_{\perp}, \mu x_3/\hbar t)\exp(i\mathbf{k}_{\perp}\cdot\mathbf{x}_{\perp})}\right|^2
\end{equation}
Integrate over spatial with equation (1.26)\footnote{Note that we only integrate over 3-axis, so the dimension is the probability per unit area.},
\[\rho_{\perp} = \int{dx_3\,\left|\braket*{\Phi_{\mathbf{x}}}{\Psi_{g}(t)}\right|^2} \rightarrow  \frac{\mu}{\hbar t(2\pi)^5}\int{dx_3\,}\left|\int{d^2k_{\perp}\,g(\mathbf{k}_{\perp}, \mu x_3/\hbar t)}\right|^2\]
\[\rightarrow\frac{1}{(2\pi)^5}\int{dk_3\,\left|\int{d^2k_{\perp}\,g(\mathbf{k}_{\perp}, \mu x_3/\hbar t)}\right|^2}\]
Hence, equation (1.23) may be written,
\begin{equation}
    \frac{dP(\hat{x}, \mathbf{k}_{0})}{d\Omega} = \rho_{\perp}\left|f_{\mathbf{k}_{0}}(\hat{x})\right|^2
\end{equation}
We define the \textit{differential cross-section} as the ratio,
\begin{equation}
    \frac{d\sigma(\hat{x}, \mathbf{k}_{0})}{d\Omega} \equiv \frac{1}{\rho_{\perp}}\frac{dP(\hat{x}, \mathbf{k}_0)}{d\Omega} = \left|f_{\mathbf{k}_0}(\hat{x})\right|^2
\end{equation}
We could think of $d\sigma(\hat{x}, \mathbf{k}_0)$ as a tiny area normal to the 3-axis, where the incoming particle must hit in order for it to be scattered into a solid angle $d\Omega$ around the direction $\hat{x}$.
The equation (1.28) then says that the probability of the above process to happen equals the ratio of $d\omega$ to the effective cross-sectional area $1/\rho_{\perp}$ of the beam.
\\\indent Of course, to measure $d\sigma/d\Omega$, experimenters do not actually send a particle or particles toward a single target. 
Instead, they direct a beam of particles toward a thin slab containing some large number $N_{\text{T}}$ of targets.
Scattering into some particular range of angles can occur only if a particle from the beam hits a tiny area $d\sigma$ around one of the targets, then the number of particles that are scattered into this range of angles\footnote{Or says, the number of scattering event.} is the number of beam particles per unit transverse area $\mathcal{N}_{\text{B}}$, times the total area $N_{\text{T}}d\sigma$ that the beam particles have to hit upon.
\section{General Scattering Theory}
There are much more general circumstances to which scattering theory is applicable.
The scattering can produce additional particles; the interaction may not be a local potential; some or all of the particles involved may be moving at relativistic velocities; some may be photon (massless); and the initial state may even contain more than two particles.
This section would describe scattering theory at a level of generality that encompasses all these possibilities.
\subsection{The S-Matrix}
We again assume that the Hamiltonian $H$ is the sum of an unperturbed Hermitian (free Hamiltonian) term $H_0$, describing any number of non-interacting particles, plus some sort of interaction $V$:
\begin{equation}
    H = H_0 + V
\end{equation}
The only assumptions we make about $V$ are that it is Hermitian, and that its effects become negligible when the particles described by $H_0$ are fall from one another\footnote{Neither scattering center nor local potential are considered in general scattering theory.}.
\\\indent In section 1.1 we defined an "in" state $\ket*{\Psi_{\mathbf{k}}^{\text{in}}}$ as an eigenstate of the Hamiltonian $H$ that looks like it consists of a single particle with momentum $\hbar\mathbf{k}$ far from the scattering center if measurements are made at sufficiently early times.
We generalize this definition, and define "in" and "out" state $\ket*{\Psi_{\alpha}^{+}}$ and $\ket*{\Psi_{\alpha}^{-}}$ as eigenstates of the Hamiltonian
\begin{equation}
    H\ket*{\Psi_{\alpha}^{\pm}} = E_{\alpha}\ket{\Psi_{\alpha}^{\pm}}
\end{equation}
and both look like an eigenstate $\Phi_{\alpha}$ of the free-particle Hamiltonian
\begin{equation}
    H_0\ket*{\Phi_{\alpha}} = E_{\alpha}\ket*{\Phi_{\alpha}}
\end{equation}
The states consist of a number of particles at great distances from each other\footnote{Where the effects of interaction are negligible.}, provided measurements are made at very early times (for $\ket*{\Psi_{\alpha}^{+}}$) or very late times (for $\ket*{\Psi_{\alpha}^{-}}$).
Here $\alpha$ is a compound index, standing for the types and numbers of the particles in the state, as well as their momenta and spin 3-componets (or helicities).
It will be convenient to choose the states $\ket*{\Phi_{\alpha}}$ to be orthonormal
\begin{equation}
    \braket*{\Phi_{\beta}}{\Phi_{\alpha}} = \delta(\beta-\alpha)
\end{equation}
The ambiguous delta function $\delta(\alpha-\beta)$ consists of a product of Kronecher deltas for the numbers and types and spin 3-componets of corresponding particles in the states $\alpha$ and $\beta$, together with 3-dimensional delta functions for the momenta of the corresponding particles in these states.
\\\indent The definition of $\ket*{\Psi_{\alpha}^{+}}$ and $\ket*{\Psi_{\alpha}^{-}}$ can be made more precise by specifying that if $g(\alpha)$ \textbf{is a sufficiently smooth function of the momenta} in the state $\alpha$, then (as a generalization of equation (1.4))
\begin{equation}
    \int{d\alpha\,g(\alpha)\exp(-iE_{\alpha}t/\hbar)\ket*{\Psi_{\alpha}^{\pm}}}\rightarrow\int{d\alpha\,g(\alpha)\exp{-iE_{\alpha}t/\hbar}\ket*{\Phi_{\alpha}}}
\end{equation}
for $t\rightarrow\mp\infty$. (Ambiguously integrals over $\alpha$ include sums over the numbers and types of particles along with the 3-components of their spins, as well as integrals over the momenta of all the particles in the state $\alpha$.)
We could satisfy this condition by rewriting equation (2.5) as a generalization of the Lippmann-Schwinger equation (1.8):
\begin{equation}
    \Psi^{\pm}_{\alpha} = \Phi_{\alpha}+\left(E_{\alpha}-H_0\pm i\epsilon\right)^{-1}V\Psi_{\alpha}^{\pm}
\end{equation}
with $\epsilon$ is positive infinitesimal parameter. Equation (2.5) then follows by a simple extension of the argument used in Section 1.1. 
From equation (2.6) we can verify the further condition (equation (2.5)) of superposition of plane wave
\[
    \int{d\alpha\,g(\alpha)\exp(-iE_{\alpha}t/\hbar)\ket*{\Psi_{\alpha}^{\pm}}} = \int{d\alpha\,g(\alpha)\exp(-iE_{\alpha}t/\hbar)\ket*{\Phi_{\alpha}}}
\]
\begin{equation}
    +\int{d\alpha}\int{d\beta\,\frac{g(\alpha)\exp(-iE_{\alpha}t/\hbar)\bra*{\Phi_{\beta}}V\ket*{\Psi_{\alpha}^{\pm}}}{E_{\alpha}-E_{\beta}\pm i\epsilon}\ket*{\Phi_{\beta}}}
\end{equation}\footnote{We integrate over $\beta$ so that the free-particle operator $H_0$ would become eigenvalue $E_{\beta}$.}
The rapid oscillation of the exponential in the second term on the right-hand side kills all contributions to this integral except those from $E_{\alpha}$ near $E_{\beta}$, where the denominator also varies rapidly. 
In particular, as we did in Section 1.1, this allows us to extend the integral to all real $E_{\alpha}$\footnote{Because $d\alpha$, as mentioned before, represents the integration over the momenta of all the particles in the state $\alpha$, which can be easily changed to integral over the energy by energy-momentum relation.}, because no part of the range of integration except very near $E_{\beta}$ will contribute as $|t|\rightarrow\infty$.
This integral can be evaluated for $|t|\rightarrow\infty$ by closing the contour of integration over $E_{\alpha}$ with a large semicircle in the upper half of the complex plane for $t\rightarrow-\infty$ or in the lower half of the complex plane for $t\rightarrow\infty$, because in both case the factor $\exp(-iE_{\alpha}t/\hbar)$ is exponentially damped on the semicircle due to the positive (negative) imagery part of $E_{\alpha}$.
In both cases the pole at $E_{\alpha} = E_{\beta}\mp i\epsilon$ is outside the contour of integration, so this integral vanishes, leaving us with equation (2.5).
\\\indent The "in" and "out" states inhabit the same Hilbert space, and are distinguished only by how they are described, by their appearance at $t\rightarrow-\infty$ or at $t\rightarrow\infty$.
Indeed, any "in" state can be expressed as a superposition of "out" states:
\begin{equation}
    \ket*{\Psi_{\alpha}^{+}} = \int{d\beta\, S_{\beta\alpha}\ket*{\Psi_{\beta}^{-}}}
\end{equation}
The coefficient $S_{\alpha\beta}$ in this relation form is known as the \textit{S-matrix}.
If we  arrange a state so that it appears at $t\rightarrow-\infty$ like a free-particle state $\ket*{\Phi_{\alpha}}$, then the state is $\ket*{\Psi_{\alpha}^{+}}$, and equation (2.8) tells us that the state will appear at late times like the superposition $\int{d\beta\,S_{\beta\alpha}\ket*{\Phi_{\beta}}}$.
As we will see, the S-matrix contains all information about the rates of reactions among particles of any sort.
\\\indent We can derive a useful formula for the S-matrix by considering what the "in" state looks like if measurements are made at \textit{late} times.
We again use equation (2.7) for $\ket*{\Psi_{\alpha}^{+}}$, but now because $t>0$ we can only close the contour of integration of $E_{\alpha}$ in the second term with a large semicircle in the \textit{lower} half of the complex plane, so now we receive a contribution from the pole at $E_{\alpha} = E_{\beta}-i\epsilon$\footnote{Because the pole is at the lower half of the complex plane.}.
Because we are integrating over a closed contour running in the \textit{clockwise} direction, the contribution of this pole is $-2\pi i$ times the same integral, but with the denominator dropped, and with the integration over $E_{\alpha}$ replaced by setting $E_{\alpha} = E_{\beta} - i\epsilon$ in the remainder of the integral. 
Because $\epsilon$ is infinitesimal positive, this is just amounts to replacing $\left(E_{\alpha}-E_{\beta}+i\epsilon\right)^{-1}$ in equation (2.7) with $-2\pi i\delta(E_{\alpha}-E_{\beta})$, so that for $t\rightarrow+\infty$
\[\int{d\alpha\,g(\alpha)\exp(-iE_{\alpha}t/\hbar)\ket*{\Psi_{\alpha}^{+}}}\rightarrow\int{d\alpha\,g(\alpha)\exp(-iE_{\alpha}t/\hbar)\ket*{\Phi_{\alpha}}}\]
\begin{equation}
    -2\pi i\int{d\alpha}\int{d\beta\,g(\alpha)\exp(-iE_{\alpha}t/\hbar)\bra*{\Phi_{\beta}}V\ket*{\Psi_{\alpha}^{+}}\delta(E_{\alpha}-E_{\beta})\ket*{\Phi_{\beta}}}
\end{equation} 
As mentioned previously, the state $\ket*{\Psi_{\alpha}^{+}}$ would, at $t\rightarrow-\infty$, look like the superposition $\int{d\beta\,S_{\beta\alpha}\ket*{\Phi_{\beta}}}$, so from equation (2.9), we have,
\begin{equation}
    S_{\beta\alpha} = \delta(\beta-\alpha)-2\pi i\delta(E_{\alpha}-E_{\beta})T_{\beta\alpha}
\end{equation} 
where
\begin{equation}
    T_{\beta\alpha} \equiv \bra*{\Phi_{\beta}}V\ket*{\Psi_{\alpha}^{+}}
\end{equation}
\\\indent By taking the scalar product of equation (2.8) with $\bra*{\Psi_{\beta}^{-}}$
\[\braket*{\Psi_{\beta}^{-}}{\Psi_{\alpha}^{+}} = \int{d\gamma\,S_{\gamma\alpha}\braket*{\Psi_{\beta}^{-}}{\Psi_{\gamma}^{-}}}\]
\[=\int{d\gamma\,S_{\gamma\alpha}\delta(\beta-\gamma)}\]
\begin{equation}
    =S_{\beta\alpha}
\end{equation}
Thus $S_{\beta\alpha}$ is the probability amplitude that a state that is arranged to look like the free-particle state $\ket*{\Phi_{\alpha}}$ at $t\rightarrow-\infty$ and would look like the free-particle state $\ket*{\Phi_{\beta}}$ when measurements are made at $t\rightarrow\infty$.
\cstars
We have chosen the free-particles states $\ket*{\Phi_{\alpha}}$ to be orthonormal, and it follows then from equation (2.6) that the "in" and "out" states are also orthonormal.
This is fairly obvious from the condition (2.5), but there is also a more direct proof.
We can evaluate the matrix element $\bra*{\Psi_{\beta^{\pm}}}V\ket*{\Psi_{\alpha}^{\pm}}$ by using equation (2.6) in either the right or left side of the scalar product.
Using the fact that $H_0$ and $V$ are Hermitian, the results must be equal applying operators on either bra or ket
\[\bra*{\Psi^{\pm}_{\beta}}V\ket*{\Phi_{\alpha}}+\bra*{\Psi^{\pm}_{\beta}}V(E_{\alpha}-H_0\pm i\epsilon)^{-1}V\ket*{\Psi^{\pm}_{\alpha}}= \]
\begin{equation}
    \bra*{\Phi_{\beta}}V\ket*{\Psi^{\pm}_{\alpha}} + \bra*{\Psi_{\beta}^{\pm}}V(E_{\beta}-H_0\mp i\epsilon)^{-1}V\ket*{\Psi_{\alpha}^{\pm}}
\end{equation}
% \footnote{The underline mean the operator is acting on the right-hand side.}
Also, we have the following identity
\[\frac{1}{E_{\alpha}-H_0\pm i\epsilon}-\frac{1}{E_{\beta}-H_0\mp i\epsilon} =\]
\[\frac{E_{\alpha}-H_0\pm i\epsilon - E_{\beta}+H_0\pm i\epsilon}{\left(E_{\alpha}-H_0\pm i\epsilon\right)\left(E_{\beta}-H_0\mp i\epsilon\right)}=\]
\begin{equation}
    \frac{E_{\alpha}-E_{\beta}\pm 2i\epsilon}{\left(E_{\alpha}-H_0\pm i\epsilon\right)\left(E_{\beta}-H_0\mp i\epsilon\right)}
\end{equation}
then divide the equation (2.13) by $E_{\alpha}-E_{\beta}\pm 2i\epsilon$
\[\frac{\bra{\Psi_{\beta}^{\pm}}V\ket*{\Phi_{\alpha}}}{E_{\alpha}-E_{\beta}\pm 2i\epsilon}-\frac{\bra*{\Phi_{\beta}}V\ket*{\Psi_{\alpha}^{\pm}}}{E_{\alpha}-E_{\beta}\pm 2i\epsilon}\]
\[=\frac{\bra*{\Psi_{\beta}^{\pm}}V(E_{\alpha}-H_0\pm i\epsilon)^{-1}V\ket*{\Psi_{\alpha}^{\pm}}-\bra*{\Psi_{\beta}^{\pm}}V(E_{\beta}-H_0\mp i\epsilon)^{-1}\ket*{\Psi_{\alpha}^{\pm}}}{E_{\alpha}-E_{\beta}\pm 2i\epsilon}\]
\begin{equation}
    =\bra*{\Psi_{\beta}^{\pm}}V(E_{\beta}-H_0\mp i\epsilon)^{-1}(E_{\alpha}-H_0\pm i\epsilon)^{-1}V\ket*{\Psi_{\alpha}^{\pm}}
\end{equation}
further simplify LHS, because the only important thing about $\epsilon$ is that it is a positive infinitesimal, so we may as well replace $2\epsilon$ here with $\epsilon$.
\[-\left(\bra{\Phi_{\alpha}}(E_{\beta}-H_0\pm i\epsilon)^{-1}V\ket*{\Psi_{\beta}^{\pm}}\right)^{*} - \bra*{\Phi_{\beta}}(E_{\alpha}-H_0\pm i\epsilon)^{-1}V\ket*{\Psi_{\alpha}^{\pm}}\]
\begin{equation}
    =\bra*{\Psi_{\beta}^{\pm}}V(E_{\beta}-H_0\mp i\epsilon)^{-1}(E_{\alpha}-H_0\pm i\epsilon)^{-1}V\ket*{\Psi_{\alpha}^{\pm}}
\end{equation}
according to equation (2.6), this tells us,
\begin{equation}
    -(\bra*{\Phi_{\alpha}}\ket*{\Psi_{\beta}^{\pm}-\Phi_{\beta}})^{*}-\braket*{\Phi_{\beta}}{\Psi_{\alpha}^{\pm}-\Phi_{\alpha}} = \braket*{\Psi_{\beta}^{\pm}-\Phi_{\beta}}{\Psi_{\alpha}^{\pm}-\Phi_{\alpha}}
\end{equation}
and therefore 
\begin{equation}
    \braket*{\Psi_{\beta}^{\pm}}{\Psi_{\alpha}^{\pm}} = \braket*{\Phi_{\beta}}{\Phi_{\alpha}} = \delta(\beta-\alpha)
\end{equation}

Because $S_{\beta\alpha}$ is the matrix of the scalar products of two complete orthonormal sets of state vectors, it must be unitary.
We can also show this directly by multiplying equation (2.13) (for "in" states)

\subsection{Rates}
\subsection{The General Optical Theorem}
\part{Quantum Field Theory}
\section{Introduction to Perturbation Theory and Scattering} 
\section{Perturbation I. Wick Diagrams}
\section{Perturbation II. Divergence and Counterterms}
\section{Feynman Diagrams}
\section{Cross-Sections S-Matrix}
\section{Computing S-Matrix Elements from Feynman Diagrams}
\end{document}